\documentclass[a4, 12pt]{article}


\usepackage{lipsum}
\usepackage[papersize={8.5in,11in}]{geometry}
\usepackage{tikz}
\usetikzlibrary{positioning,shapes,shadows,arrows}

\RequirePackage[T1]{fontenc} \RequirePackage[tt=false, type1=true]{libertine} \RequirePackage[varqu]{zi4} \RequirePackage[libertine]{newtxmath}
\usepackage{spverbatim}
\usepackage{minted}
\usepackage{xcolor}
\definecolor{frenchblue}
{rgb}{0.6, 0.1, 0.95}
\definecolor{electricindigo}{rgb}{0.1, 0.05, 0.9}
\definecolor{coral}{rgb}{0.2, 0.5, 0.5}
\definecolor{lightcarminepink}{rgb}{0.9, 0.4, 0.38}
\definecolor{ceruleanblue}{rgb}{0.5,0.52, 0.98}
\definecolor{flowerblue}{rgb}{0.39, 0.58, 0.93}
\usepackage[colorlinks = true,
            linkcolor = electricindigo,
            urlcolor  = blue,
            citecolor = blue,
            anchorcolor = blue]{hyperref}

\usepackage{graphicx}
\usepackage{setspace}
\usepackage{amsthm}
\newtheorem{definition}{Definition}
\newtheorem{theorem}{Theorem}
\newtheorem{claim}{Claim}
\newtheorem{lemma}{Lemma}
\newtheorem{observation}{Observation}
\newtheorem{proposition}{Proposition}
\newtheorem{corollary}{Corollary}
\usepackage{amsmath}
\usepackage{amsfonts}
\usepackage[capitalize]{cleveref}
 \setcounter{tocdepth}{4} \setcounter{secnumdepth}{4}
 
\newcommand{\josh}[1]{{\color{blue}{\bf{J: #1}}}}

\newcommand{\g}[1]{{\color{electricindigo}{{G: #1}}}}

\title{Sunflowers and $$\mathsf{AC}^0$$ circuits}
\author{Gülce Kardeş
}
\date{\today}

\usepackage{hyperref}  

\begin{document}
\maketitle
Recently, I worked on an exciting formalization project that bridges combinatorics and computational complexity via an elegant use of the Erdős-Rado Sunflower Theorem from 1960.
My focus in this project has been on implementing a breakthrough from a 1995 paper by Håstad, Jukna, and Pudlák \cite{Hstad1995}.

This work introduced a novel "top-down" technique that uses sunflower structures to establish lower bounds on circuit complexity. The innovation lies in the use of "limit vectors" that force small circuits to make mistakes, whose existence we can prove via the existence of ("parity") sunflowers in the hypercube. By showing the existence of such sunflowers, the authors proved that depth-3 $$\mathsf{AC}^0$$ circuits (poly-size, constant-depth, unbounded fan-in boolean circuit using $$\mathsf{AND}$$, $$\mathsf{OR}$$ gates and input-level $$\mathsf{NOT}$$ gates, arranged in alternating levels where each level consists entirely of one gate type, $$\mathsf{AND}$$ or $$\mathsf{OR}$$) computing parity require size at least $$2^{0.61\sqrt{n}}$$, while majority requires $$2^{0.849\sqrt{n}}$$. 

Let us introduce the key definitions to outline the proof structure.
\begin{definition}
Let $$B \subseteq\{0,1\}^n$$ be a set of vectors. $$A$$ vector $$y \in\{0,1\}^n$$ is a $$k$$-limit for a set $$B$$ if, for any subset of indices $$S \in[n]^k$$, there exists a vector $$x \in B$$ such that $$x \neq y$$ and $$\left.y\right|_s=\left.x\right|_s$$. If $$x>y$$ instead of $$x \neq y$$, we call $$y$$ a lower $$k$$-limit for $$B$$.  
\end{definition} 

\begin{definition}
A circuit $$C$$ separates the pair $$(A, B)$$ if $$C$$ computes 1 (resp. 0) on inputs from $$A$$ (resp. $$B$$). 
\end{definition}
A depth-3 $$\mathsf{AC}^0$$ circuit attempting to distinguish inputs of odd parity (set $$A$$) from those of even parity (set $$B$$) must be robust against (lower) $$k$$-limits. In other words, the circuit must correctly classify strings in $$A$$ even when they locally look very similar to strings in $$B$$. The paper's key lemma proves such robustness is impossible for depth-3 $$\mathsf{AC}^0$$ circuits. Sunflowers enter naturally here - their existence in the hypercube implies the existence of these troublesome $$k$$-limit inputs that force the circuit to fail. 
\begin{definition}
A family of sets $$X_1, \ldots, X_{k+1}$$ is a sunflower with $$(k+1)$$ petals and core $$Y$$ if:
all sets $$X_i$$ have the same size $$s$$ (uniformity),
for every $$i \neq j$$, $$X_i \cap X_j = Y$$ (common intersection), and the sets $$X_i \setminus Y$$ (the petals) are pairwise disjoint.
\end{definition}
"When are sunflowers inevitable?", you might ask. The Erdős-Rado theorem ensures that such structures emerge in any sufficiently large family of bounded-size sets. You might think of it as structure from randomness, and the key intuition comes from the pigeonhole principle: when you have a lot of sets of bounded size, there can only be so much "variety" in how they intersect. At some point, you're forced to have sets that intersect in a very structured way. The proof of the theorem turns this intuition into mathematics through careful application of the pigeonhole principle combined with induction.
\begin{theorem}
(Erdős \& Rado 1960). Let $$\mathcal{F}$$ be a family with more than $$s!(k-1)^s$$ sets of cardinality $$s$$. Then $$\mathcal{F}$$ contains a sunflower with $$k$$ petals.
\end{theorem}
\begin{proof}
     Induction on the set size $$s$$ in our family $$\mathcal{F}$$. For the base case $$s=1$$, since $$|\mathcal{F}| > (k-1)$$, any $$k$$ single-element sets automatically form a sunflower with $$k$$ petals and empty core, as single points are disjoint. For the inductive step $$s \geq 2$$, we take a maximal family $$\mathcal{A} = \{A_1, ..., A_t\}$$ of pairwise disjoint sets from $$\mathcal{F}$$. If $$t \geq k$$, these sets form our desired sunflower with empty core. Otherwise, for $$t < k$$, let $$B = \bigcup_{i=1}^t A_i$$ be the union of sets in $$\mathcal{A}$$. Since $$\mathcal{A}$$ was maximal, $$B$$ intersects every set in $$\mathcal{F}$$, and $$|B| \leq s(k-1)$$. By the pigeonhole principle, some point $$x \in B$$ must appear in at least $$\frac{|\mathcal{F}|}{|B|} > \frac{s!(k-1)^s}{s(k-1)} = (s-1)!(k-1)^{s-1}$$ sets of $$\mathcal{F}$$. We can then apply our inductive hypothesis to the family $$\mathcal{F}_x = \{S \setminus {x} : S \in \mathcal{F}, x \in S\}$$, obtaining a sunflower with $$k$$ petals. Adding $$x$$ back to each set in this sunflower yields our desired sunflower in $$\mathcal{F}$$, as the size reduction preserves enough sets for the induction while dropping the set size by 1.
\end{proof}
Using Coq, I've started to build a formal verification of the circuit size lower bounds of \cite{Hstad1995} from first principles. While the proof's foundation is straightforward - basic set operations and $$\mathsf{AC}^0$$ circuit definitions, and even some of the lemmas leading to the sunflower theorem - the real challenge emerged in formalizing the existence of sunflowers, i.e., the proof above. 

Several auxiliary lemmas in the proof of the sunflower theorem remain admitted, and this is where you might come in. I welcome collaboration on completing these remaining pieces. Meanwhile, I'll be using pen and paper to explore some other combinatorial structures and how they might be intertwined with circuits.
\bibliographystyle{IEEEtran}
\bibliography{references.bib}
\end{document}
